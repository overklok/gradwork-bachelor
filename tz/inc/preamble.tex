%%% Преамбула %%%

\documentclass[oneside,final,14pt]{extreport}
\usepackage[utf8]{inputenc}
\usepackage[english,russian]{babel}
\usepackage{vmargin}
\setpapersize{A4}
\setmarginsrb{25mm}{20mm}{10mm}{20mm}{0pt}{0mm}{0pt}{13mm}
\usepackage{indentfirst}

\usepackage{misccorr}
\usepackage{graphicx}
\usepackage{amsmath}
\usepackage{amsthm}

\usepackage{tabularx}

\usepackage{setspace}

\linespread{1.5}

% Оформление заголовков
\usepackage{titlesec}

\setcounter{secnumdepth}{3}

\titleformat{\chapter}[block]{\normalfont\Large\bfseries}{\thechapter}{10pt}{}
\titleformat{\section}[block]{\normalfont\large\bfseries}{\thesection}{10pt}{}
\titleformat{\subsection}[block]{\normalfont\bfseries}{\thesubsection}{10pt}{}
\titleformat{\subsubsection}[block]{\normalfont\bfseries}{\thesubsubsection}{10pt}{}

\titlespacing*{\chapter}{\parindent}{0pt}{0pt}
\titlespacing*{\section}{\parindent}{0pt}{0pt}
\titlespacing*{\subsection}{\parindent}{0pt}{0pt}
\titlespacing*{\subsubsection}{\parindent}{0pt}{0pt}

\titleformat{\bibliography}[block]{\normalfont\large\bfseries}{\thesection}{10pt}{} % заголовок библиографии

\renewcommand{\thechapter}{\arabic{chapter}}
\renewcommand{\thesection}{\arabic{chapter}.\arabic{section}}
\renewcommand{\thesubsection}{\arabic{chapter}.\arabic{section}.\arabic{subsection}}
\renewcommand{\thesubsubsection}{\arabic{chapter}.\arabic{section}.\arabic{subsection}.\arabic{subsubsection}}

% Оформление заголовка оглавления
\usepackage{tocloft}

\renewcommand{\cfttoctitlefont}{\hfil \hbox{} \bfseries \Large} % по центру

\setlength{\cftbeforetoctitleskip}{0pt} % отступ до заголовка
\setlength{\cftaftertoctitleskip}{0pt} % отступ после заголовка

% Оформление пунктов оглавления
\renewcommand{\cftchapfont}{\normalfont} % убрать полужирное начертание глав
\renewcommand{\cftchappagefont}{\normalfont} % убрать полужирное начертание страниц глав
\setlength{\cftbeforechapskip}{0pt} % убрать вертикальные отступы глав
\renewcommand{\cftchapleader}{\cftdotfill{\cftdotsep}} % точечные линии для глав

% Отступы элементов списков
\usepackage{enumitem}
\setlist[itemize]{noitemsep, nolistsep}
\setlist[enumerate]{noitemsep, nolistsep}

% Нумерация списков
\renewcommand\labelenumi{\arabic{enumi}}
\renewcommand\labelenumii{\theenumi.\arabic{enumii}}

% Маркеры списков
\renewcommand\labelitemi{---}

% Абзац
\parindent=1.27cm % абзацный отступ

% Изменение автоматических подписей (кроме списка литературы)
\addto\captionsrussian{
	\renewcommand{\contentsname}{Содержание}
	\def\figurename{\normalfont Рисунок}
	\def\tablename{\normalfont Таблица}
}

% Подписи
\usepackage[margin=10pt,font=normal,labelfont=bf,labelsep=endash]{caption} %подпись к рисункам
\captionsetup[figure]{justification=centering}
\captionsetup[table]{justification=raggedright,singlelinecheck=off}

% Таблицы

\usepackage{tabularx}

% Список литературы
\makeatletter
	\renewcommand{\@biblabel}[1]{#1.}
\makeatother

% Гиперссылки
\usepackage{hyperref}

% Символ №
\usepackage{textcomp}
\newcommand*{\No}{\textnumero}

% Прочее оформление
\usepackage[usenames]{color} % задание цвета текста и фона
\usepackage{colortbl} % задание цвета таблицы

\sloppy