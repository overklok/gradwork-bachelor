\chapter{Требования к системе}

\section{Требования к системе в целом}

\subsection{Требования к структуре и функционированию}

\par
	Основной состав функций API САТУ:

	\begin{itemize}
		\item приём входящих данных об объектах анализа и приведение этих данных к единому формату;
		\item проверка данных на корректность (валидация);
		\item вызов процедур анализа проверенных данных;
		\item авторизация пользователей, загружающих данные;
		\item отправка отчётов о статусе принятых данных клиентским приложениям;
		\item предоставление результатов анализа клиентским приложениям.
	\end{itemize}
	
	API САТУ должна поддерживать принцип модульности:

	\begin{itemize}
		\item относительно клиентских приложений - легко и без непредвиденных побочных эффектов соединяться и работать с ними вне зависимости от их количества. 
		\item относительно системы в целом - выход API из строя (при больших вычислительных загрузках или сетевых атаках) не должен влиять на работу остальных компонентов САТУ, включая СУБД.
	\end{itemize}
 
	Структура API должна быть устойчива к появлению в системе новых компонентов таким образом, чтобы реализация новых команд в интерфейсе не создавала необходимости в его повторном проектировании.

\subsubsection{Перечень подсистем, их назначение и основные характеристики}

\par

	Состав подсистем API САТУ:

	\begin{enumerate}
		\item подсистема приёма запросов и отправки данных; \label{tz:subsysreq:http}
		\item подсистема аутентификации пользователей; \label{tz:subsysreq:auth}
		\item подсистема формирования запросов к БД; \label{tz:subsysreq:db}
		\item подсистема валидации данных; \label{tz:subsysreq:validate}
		\item подсистема унификации и форматирования данных. \label{tz:subsysreq:uniform}
	\end{enumerate}

	Подсистема \ref{tz:subsysreq:http} предназначена для взаимодействия с клиентскими приложениями по сети Интернет и содержит процедуры обработки исходящих HTTP-запросов, в том числе для аутентификации пользователей этих приложений и получения снимков, а также для отправки отчётов и результатов анализа.

	Подсистема \ref{tz:subsysreq:auth} контролирует подлинность запросов от клиентских приложений, тем самым исключая получение данных от злоумышленников и несанкционированных пользователей. Механизм поддерживает использование нескольких факторов аутентификации, в частности, для пользователей мобильных приложений может применяться дополнительная аутентификация по номеру телефона (кроме пароля).

	Подсистема \ref{tz:subsysreq:db} принимает участие в обработке клиентских запросов (работа подсистемы \ref{tz:subsysreq:http}), требующих доступа к БД и генерирует запросы к СУБД для загрузки (извлечения) данных непосредственно в (из) БД.

	Подсистема \ref{tz:subsysreq:db} отвечает за корректность данных, поступающих в модуль работы с ИК изображениями САТУ и впоследствии загружаемых в БД, путём проверки наличия основных характеристик допустимых форматов данных и свойств файлов, исключая таким образом разного рода инъекции (программные или по содержимому), а также данные низкого качества.

	Подсистема \ref{tz:subsysreq:uniform} устраняет различия в форматах данных во входящих пакетах путём преобразования файлов одного стандарта в принятый САТУ стандарт (такой, как \texttt{JPEG} или \texttt{TIFF}), а также преобразования форматов некоторых алфавитно-цифровых полей, совместимых с форматами соответствующих полей в СУБД (например, \textit{форматы хранения даты-времени, координат}).

\subsubsection{Требования к характеристикам взаимосвязей создаваемой системы со смежными подсистемами}

\par

	Клиентские приложения (в том числе мобильное приложение), реализуемые сторонними разработчиками, должны соблюдать структуру запросов согласно прилагаемой документации для разработчиков.

\subsection{Надёжность}

\par

	API играет роль приёмника новых данных для САТУ, поэтому должен обеспечить максимальную надёжность информации, приходящей от внешних запросов, а именно:

	\begin{itemize}
		\item исключить из обработки неполные пакеты данных, данные низкого качества, а также данные, не содержащие основные признаки, характеризующие конкретный формат (например, тепловой снимок выдан за фотоснимок и прочие инъекции файлов);
		\item проводить аутентификацию пользователей клиентских приложений (через пароль / SMS-подтверждение), а также самих клиентских приложений (путём использования специальных ключей доступа API);
		\item блокировать запросы от пользователей, намеренно отправляющих некорректные данные (занесение в чёрный список).
	\end{itemize}
	
	Не менее важен вопрос надёжности функционирования, поскольку серверы API должны работать в постоянном режиме: web-серверы, предназначенные для работы API, должны быть оснащены программными фильтрами DDoS-атак, контролирующими число запросов отдельных пользователей и блокирующие их в случае превышения этого числа.

\subsection{Защита от несанкционированного доступа}

\par

	Защита от НСД в API САТУ должна проводиться на двух уровнях: 

	\begin{itemize}
		\item аутентификация клиентских приложений;
		\item аутентификация пользователей клиентских приложений.
	\end{itemize}
 
	Поскольку API САТУ подразумевает открытую работу в глобальной сети, то следует учитывать, что в роли злоумышленника могут выступать не только пользователи, загружающие данные, но и программы, отправляющие эти данные от имени представившихся им пользователей. Поэтому разработчики клиентских приложений должны зарегистрировать свои программы в системе, чтобы API мог контролировать их активность и блокировать нежелательные приложения в случае обнаружения угрозы. Таким образом, каждое приложение должно обладать специальным ключом доступа (токеном), который выдаётся разработчику для использования в приложении, чтобы оно могло формировать правильные запросы к API.
	Пользователи таких приложений могут регистрироваться, а затем аутентифицироваться в системе, используя клиентские приложения, обладающие ключом доступа. При регистрации система получает данные о пользователе, включая пароль в зашифрованном виде. Впоследствии аутентификация производится по стандартной схеме парольной защиты.

\section{Требования к видам обеспечения}

\subsection{Информационное обеспечение}

\par

	\textbf{API САТУ} выполняет обмен информацией между следующими подсистемами САТУ: передача обработанных снимков и результатов анализа \textbf{web-приложению}; сформированные запросы к БД и проверенные снимки передаются \textbf{СУБД}.
 
	СУБД, работающая в API, должна иметь доступ к базе данных САТУ, которая хранит результаты анализа, обработанные снимки зданий, дополнительную информацию о съёмке и самих зданиях. Кроме того, структура базы данных должна учитывать возможность хранения информации о самих пользователях, в том числе данные для аутентификации. СУБД в API должны быть доступны операции сохранения, изменения и удаления данных.
 
	Кроме того, API использует данные сторонних ГИС о местоположении объектов при добавлении их в базу. Это необходимо для приложений, которые используют API для визуализации данных на карте.

	{\color{blue} Дополнить про Интернет}

\subsection{Программное обеспечение}

\par

	Для работы API в сети требуется следующее программное обеспечение:

	\begin{itemize}
		\item web-сервер \texttt{nginx 1.13.x} / \texttt{Apache HTTP Server 2.4.x};
		\item СУБД \texttt{MySQL 5.6} и выше;
		\item интерпретатор языка программирования \texttt{PHP 5.6} и выше:
		\item операционная система \texttt{FreeBSD} или семейства \texttt{Linux}.
	\end{itemize}
	
	Операционная система и средства защиты от сетевых атак должны поставляться и поддерживаться облачным PaaS-сервисом.
	
\par
	Для разработки, тестирования и развёртывания API используется:

	\begin{itemize}
		\item среда разработки \texttt{IntelliJ IDEA};
		\item система контроля версий \texttt{git};
		\item средство отладки и тестирования API \texttt{Postman}.
	\end{itemize}
 
	Для работы с API клиентские приложения также должны иметь доступ к Интернет.

\subsection{Техническое обеспечение}

\par

	Минимальные требования к серверной ЭВМ:

	\begin{itemize}
		\item Частота ЦП >= 2-3 ГГц, 
		\item Объём оперативной памяти >= 8192 Мб,
		\item ПЗУ - жёсткие диски (основной и резервный) >= 1 Тб,
		\item Сетевой адаптер - Ethernet, гигабитный, скорость шины 10 Гбит/с
	\end{itemize}

\subsection{Методическое обеспечение}

\par

	Вместе с API САТУ должна поставляться документация для разработчиков клиентских приложений, описывающая набор команд, структуру их параметров и требования к данным, а также описание процесса работы с командами и примеры их использования. Кроме того, в документации должен содержаться раздел, описывающий процесс регистрации приложений в API.

