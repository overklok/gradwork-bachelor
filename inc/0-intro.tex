{
	\titleformat{\section}[block]{\centering\normalfont\large\bfseries}{\thesection}{10pt}{}
	\section*{\centering Введение}
}
\addcontentsline{toc}{chapter}{Введение}

\par
	Согласно аналитическим данным, в России расход тепловой энергии на отопление многоквартирных домов составляет примерно 45\% от объёма всех энергетических ресурсов страны \cite{intro:thermo-analysis-federal}. Такое повышенное по сравнению с мировым значение можно объяснить тем, что большая часть территории страны расположена в северных областях, характеризующихся холодным климатом. По оценкам, проводимым организацией МЭА, в мировом масштабе процент потребления тепловой энергии для отопления зданий составляет 32-33\% \cite{intro:heating-drivers}. Прогноз этого показателя в долгосрочной перспективе показывает его рост в ближайшие несколько десятков лет при большинстве сценариев увеличения суммарной площади помещений \cite{intro:energy-use-perspective}.

\par
	В связи с этим актуальна задача повышения энергоэффективности зданий. Существует ряд причин возникновения тепловых потерь в помещениях, в частности имеют место тепловые утечки, происходящие в различных областях помещений (стены, крыши, окна, элементы вентиляции и пр.) \cite{intro:heat-loss-sources, intro:sources-residental}.

\par
	Для решения этой задачи применяют различные методы анализа теплопотерь. Наиболее распространенным на сегодняшний день считается использование тепловизионного сканирования зданий, которое позволяет не только дать точную оценку уровня тепловых утечек, но и обнаружить источники самих теплопотерь. Существует большое число частных компаний, занимающихся тепловизионным анализом помещений и выполняющих эту работу вручную. В этой связи возникла тенденция к созданию автоматизированных систем контроля тепловых утечек в городских зданиях. Потенциал таких систем, несомненно, высок, поскольку они способны накапливать в себе большые объёмы данных, которые можно подвергнуть статистическому анализу и на этой основе получать полезную информацию для частных владельцев домов, коммунальных организаций и городских служб.

\par
	Целью данной работы является проведение анализа имеющихся в мире систем контроля тепловых утечек, выбор прототипного решения среди этих систем и разработка улучшенной системы на его основе путём исправления выявленных недостатков.
