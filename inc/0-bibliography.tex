\begin{thebibliography}{00}
	\bibitem{intro:thermo-analysis-federal}
	\emph{Анализ потребления тепловой энергии на отопление многоквартирных домов как способ повышения энергоэффективности в сфере ЖКХ} [Электронный ресурс] / Дирекция по проблемам ЖКХ //
	Аналитический центр при Правительстве Российской Федерации. --- 2013. --- Режим доступа:
	\url{http://gkh-altay.ru/d/205499/d/06_24_kr_stol_analitika_dor_abotannaya_po_rezultata_m.pdf} (дата обращения: 29.03.2017).

	\bibitem{intro:heating-drivers}
	\emph{Heating and cooling energy trends and drivers in buildings} [Текст] / Ürge-Vorsatz D., Cabeza L. F., Serrano S., Barreneche C., Petrichenko K. //
	Renewable and Sustainable Energy Reviews --- Budapest, Hungary: Elsevier, 2015. --- \No{} 41 --- С. 85-98.

	\bibitem{intro:heat-loss-sources}
	\emph{What are the sources of home heat loss?} [Электронный ресурс] / Wilson L. //
	Shrink That Footprint --- Режим доступа: \url{http://shrinkthatfootprint.com/home-heat-loss} (дата обращения: 29.03.2017).

	\bibitem{intro:sources-residental}
	\emph{Detecting sources of heat loss in residential buildings from infrared imaging} [Электронный ресурс] / Chen S., Chen E. //
	Massachusetts Institute of Technology. Dept. of Mechanical Engineering --- 2011. --- Режим доступа: \url{http://hdl.handle.net/1721.1/68921} (дата обращения: 20.02.2017).

	\bibitem{intro:energy-use-perspective}
	\emph{Energy use in buildings in a long-term perspective} [Текст] / Ürge-Vorsatz D., Petrichenko K., Staniec M., Eom J. //
	Current Opinion in Environmental Sustainability --- Budapest, Hungary: Elsevier, 2013. --- \No{} 5 --- С. 141-151.
\end{thebibliography}