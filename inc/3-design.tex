\chapter{Проектирование предлагаемого решения}

\section{Внешнее проектирование}

\par
	В результате внешнего проектирования было получено техническое задание по ГОСТ 34.602-89 на разработку программного интерфейса (API) искомой системы, приведенное в приложении *.

\section{Внутреннее проектирование}

\par

	На основе требований, выдвинутых в техническом задании и результатов моделирования, приведённых в главе \ref{chap:models} проведено внутреннее проектирование API САТУ.

	Согласно требованиям, API САТУ взаимодействует с клиентскими приложениями через Интернет. В этой связи разумно использовать инструменты, ориентированные на работу в Web.

	Для разработки программного интерфейса системы анализа тепловых утечек в городских зданиях был выбран сценарный язык программирования PHP, поддерживающий концепцию объектно-ориентированного программирования.
	% Основной причиной выбора этого языка является его абсолютное лидерство при использовании в web-проектах, в том числе для web-API, обоснованное наличием большого количества соответствующих встроенных средств, ориентированных на разработку web-проектов.

\par

	Проектирование API выполнено на основе шаблона \texttt{MVC (Model-View-Controller)}, который подразумевает деление программного кода на 3 отдельных компонента: модель (отвечает за обработку данных), представление (отвечает за получение и отправление данных) и контроллер (содержит управляющую логику, использует модели и представления). MVC позволяет сделать независимым код, отвечающий за работу с данными от кода, отвечающего за приём и отправку запросов.
	В качестве схемы взаимодействия клиентских приложений и компонентов системы, работающими на серверах, была взята архитектура \texttt{REST}. Ограничения, определённые в REST, позволяют создавать масштабируемые и унифицированные программные интерфейсы [*]. \\

	Стандарт, который использует API для форматирования сообщений - \texttt{JSON}. Формат JSON подходит для передачи сообщений со сложной структурой и уместен при обмене данными между клиентскими приложениями и сервером.
	Все вышеуказанные инструменты охвачены программной платформой \texttt{Yii 2 Framework}, полностью реализующей указанные архитектуры и стандарты. Данная платформа позволяет ускорить процесс разработки web-проектов, в том числе и API. В процессе внутреннего проектировании предлагаемого решения выяснилось, что эта платформа является наиболее подходящей.

\section{Результаты и выводы по главе 3}

\par

	\textcolor{red}{Раздел в доработке.}